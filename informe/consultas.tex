\begin{section}{Implementaci\'on}

Para la resoluci\'on de este trabajo se decidi\'o utilizar el motor de base de datos $SQLite$ como base para su implementaci\'on dado que es m\'as liviano para correr y su uso es bastante intuitivo. Puede soportar una cantidad abundate de datos (aproximadamente 2 teras) y tambi\'en utiliza llamadas simples a subrutinas y funciones lo que reduce la latencia en el acceso a la base de datos, debido a que las llamadas a funciones son más eficientes que la comunicación entre procesos. 

Para la confecci\'on de las tablas realizamos un script de python que hace uso de varios csvs para llenar las tablas. 

\begin{lstlisting}[language=SQL]



\end{lstlisting} 

\end{section}

\begin{section}{Consultas}


\noindent Consulta 1: Obtener los ganadores de las elecciones transcurridas en el \'ultimo año.

\begin{lstlisting}[language=SQL, basicstyle=\footnotesize]

SELECT v.nombre
FROM Votante v, Candidato can, VotacionCandidato vc, SePostulaA spa
WHERE v.dni = can.dni
AND can.dni = spa.dni
AND vc.idEleccion IN (SELECT vc2.idEleccion
                      FROM VotacionEleccion ve, VotacionCandidato vc2
                      WHERE ve.idEleccion = vc2.idEleccion
                      AND fecha >= ( SELECT MAX(fecha)
                                     FROM VotacionEleccion))
AND vc.idEleccion = spa.idEleccion
AND spa.cantVotos = (SELECT MAX(spa2.cantVotos)
                     FROM SePostulaA spa2
                     WHERE spa2.idEleccion = vc.idEleccion);


\end{lstlisting} 
Esta consulta toma los candidatos que participan de las \'ultimas elecciones a partir de la VotacionCandidato y luego los filtra de acuerdo con la cantidad de votos, para as\'i obtener a los ganadores.

~

\noindent Consulta 2: Consultar las 5 personas que m\'as tarde fueron a votar antes de terminar la votaci\'on por cada centro electoral en na elecci\'on.

\begin{lstlisting}[language=SQL, basicstyle=\footnotesize]

SELECT cen.direccion, vot.dni
FROM Vota v, Votante vot, Mesa m, Centro cen
WHERE v.idMesa = m.idMesa
AND vot.dni = v.dni
AND m.idCentro = cen.idCentro
AND v.idEleccion IN (SELECT ve.idEleccion
                     FROM VotacionEleccion ve
                     WHERE fecha >= (SELECT MAX(fecha)
                                     FROM VotacionEleccion))
AND vot.dni IN (SELECT vot1.dni
                FROM Votante vot1, Vota v1, Mesa m1
                WHERE v1.idMesa = m1.idMesa
                AND m1.idCentro = cen.idCentro
                AND vot1.dni = v1.dni
                ORDER BY hora DESC
                LIMIT 5)
ORDER BY direccion,hora DESC;

\end{lstlisting} 

Aqu\'i nos basamos en la \'ultima elecci\'on para seleccionar a los votantes que participaronde ella y luego tomamos \'unicamente aquellos que en horario llegaron m\'as tarde. El ORDER BY hora DESC sumado al LIMIT 5 del SELECT anidado nos permite restringir los DNIs de los votantes por horario m\'as tard\'io. Luego, agrupamos las tuplas de salida por centro para clarificar el resultado.

~

\noindent Consulta 3: Consultar quienes fueron los partidos pol\'iticos que obtuvieron m\'as del 20$\%$ en las \'ultimas 5 elecciones provinciales a gobernador.



\begin{lstlisting}[language=SQL, basicstyle=\footnotesize]

SELECT pp.nombre
FROM PartidoPolitico pp, 
     SePostulaA spa, 
     Candidato can, 
     Cargo c, 
     VotacionCandidato vc
WHERE pp.idPartido = spa.idPartido
AND can.dni = spa.dni
AND spa.idEleccion = vc.idEleccion
AND vc.idEleccion IN (SELECT ve.idEleccion
                      FROM VotacionCandidato vc2, 
                           VotacionEleccion ve, Cargo c2
                      WHERE vc2.idEleccion = ve.idEleccion
                      AND c2.idCargo = vc2.idCargo
                      AND c2.nombre LIKE 'Gobernador'
                      ORDER BY ve.fecha DESC
                      LIMIT 5)
AND vc.idCargo = c.idCargo
AND c.nombre LIKE 'Gobernador'
AND spa.cantVotos > (SELECT SUM(spa2.cantVotos)*0.2
                     FROM SePostulaA spa2
                     WHERE spa2.idEleccion = vc.idEleccion);
\end{lstlisting} 

En este caso tomamos los partidos pol\'iticos de los candidatos que participan en las \'ultimas cinco elecci\'ones de manera similar a como tomamos los votantes m\'as tard\'ios de la consulta anterior, filtrando las elecciones por cargo a Gobernador. Luego pedimos que dichos candidatos junten una cantidad de votos mayor al 20\% de la suma total de votos en la elecci\'on de la cual participan.

~ 


\end{section}
