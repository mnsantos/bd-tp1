\begin{section}{Implementaci\'on}

Para la resoluci\'on de este trabajo se decidi\'o utilizar el motor de base de datos $SQLite$ como base para su implementaci\'on dado que es m\'as liviano para correr y su uso es bastante intuitivo. Puede soportar una cantidad abundate de datos (aproximadamente 2 teras) y tambi\'en utiliza llamadas simples a subrutinas y funciones lo que reduce la latencia en el acceso a la base de datos, debido a que las llamadas a funciones son más eficientes que la comunicación entre procesos. 

Para la confecci\'on de las tablas realizamos un script de python que hace uso de varios csvs para llenar las tablas.
A continuaci\'on podemos observar el esquema de las tablas creadas:

\begin{subsection}{Esquemas}
\begin{lstlisting}[language=SQL, basicstyle=\footnotesize]

CREATE TABLE Cargo(
  idCargo INTEGER PRIMARY KEY, 
  nombre VARCHAR2, 
  fechaInicio INTEGER, 
  fechaFin INTEGER,
  CHECK (fechaFin = -1 OR fechaFin > fechaInicio)
);

CREATE TABLE Territorio(
  idTerritorio INTEGER PRIMARY KEY, 
  nombre VARCHAR2, 
  idSupraTerritorio INTEGER, 
  FOREIGN KEY(idSupraTerritorio) REFERENCES Territorio(idTerritorio)
);

CREATE TABLE Votante(
  DNI INTEGER PRIMARY KEY, 
  nombre VARCHAR2, 
  idTerritorio INTEGER, 
  FOREIGN KEY(idTerritorio) REFERENCES Territorio(idTerritorio)
);

CREATE TABLE Candidato(
  DNI INTEGER PRIMARY KEY,
  FOREIGN KEY(DNI) REFERENCES Votante(DNI)
);

CREATE TABLE Camioneta(
  Patente VARCHAR2 PRIMARY KEY, 
  DNI INTEGER, 
  FOREIGN KEY(DNI) REFERENCES Votante(DNI)
);

CREATE TABLE Centro(
  idCentro INTEGER PRIMARY KEY, 
  direccion VARCHAR2, 
  patente VARCHAR2, 
  idTerritorio INTEGER, 
  FOREIGN KEY(patente) REFERENCES Camioneta(patente), 
  FOREIGN KEY(idTerritorio) REFERENCES Territorio(idTerritorio)
);

CREATE TABLE Mesa(
  idMesa INTEGER PRIMARY KEY, 
  idCentro INTEGER, 
  FOREIGN KEY(idCentro) REFERENCES Centro(idCentro)
);

CREATE TABLE VotacionPorMesa(
  idEleccion INTEGER, 
  idMesa INTEGER, 
  DNI_Presidente INTEGER, 
  DNI_Vice INTEGER, 
  DNI_Tecnico INTEGER, 
  PRIMARY KEY(idEleccion,idMesa), 
  FOREIGN KEY(idEleccion) REFERENCES VotacionEleccion(idEleccion), 
  FOREIGN KEY(idMesa) REFERENCES Mesa(idMesa), 
  FOREIGN KEY(DNI_Presidente) REFERENCES Votante(DNI), 
  FOREIGN KEY(DNI_Vice) REFERENCES Votante(DNI), 
  FOREIGN KEY(DNI_Tecnico) REFERENCES Votante(DNI)
);

CREATE TABLE EsFiscal(
  DNI INTEGER, 
  idEleccion INTEGER, 
  idMesa INTEGER, 
  idPartido INTEGER, 
  PRIMARY KEY(DNI,idEleccion,idMesa), 
  FOREIGN KEY(idEleccion) REFERENCES VotacionEleccion(idEleccion), 
  FOREIGN KEY(idMesa) REFERENCES Mesa(idMesa), 
  FOREIGN KEY(DNI) REFERENCES Votante(DNI), 
  FOREIGN KEY(idPartido) REFERENCES PartidoPolitico(idPartido)
);

CREATE TABLE VotacionEleccion(
  idEleccion INTEGER PRIMARY KEY, 
  tipo INTEGER,
  fecha INTEGER,
  CHECK (tipo = 1 OR tipo = 2)
);

CREATE TABLE VotacionCandidato(
  idEleccion INTEGER PRIMARY KEY, 
  idCargo INTEGER, 
  idTerritorio INTEGER, 
  FOREIGN KEY(idEleccion) REFERENCES VotacionEleccion(idEleccion), 
  FOREIGN KEY(idCargo) REFERENCES Cargo(idCargo), 
  FOREIGN KEY(idTerritorio) REFERENCES Territorio(idTerritorio)
);

CREATE TABLE VotacionConsultaPopular(
  idEleccion INTEGER PRIMARY KEY,
  FOREIGN KEY(idEleccion) REFERENCES VotacionEleccion(idEleccion)
);

CREATE TABLE Vota(
  DNI INTEGER, 
  idEleccion INTEGER, 
  idMesa INTEGER, 
  hora INTEGER, 
  PRIMARY KEY(DNI,idEleccion,idMesa), 
  FOREIGN KEY(DNI) REFERENCES Votante(DNI), 
  FOREIGN KEY(idEleccion) REFERENCES VotacionEleccion(idEleccion), 
  FOREIGN KEY(idMesa) REFERENCES Mesa(idMesa),
  CHECK (hora >= 0 AND hora <= 23)
);

CREATE TABLE PartidoPolitico(
  idPartido INTEGER PRIMARY KEY, 
  nombre VARCHAR2
);

CREATE TABLE ConsultaPopular(
  idConsulta INTEGER PRIMARY KEY,
  idEleccion INTEGER,
  descripcion VARCHAR2,
  FOREIGN KEY(idEleccion) REFERENCES VotacionConsultaPopular(idEleccion)
);

CREATE TABLE Voto(
  idVoto INTEGER PRIMARY KEY ASC,
  idEleccion INTEGER,
  idMesa INTEGER,
  tipo INTEGER,
  FOREIGN KEY(idEleccion) REFERENCES VotacionEleccion(idEleccion),
  FOREIGN KEY(idMesa) REFERENCES Mesa(idMesa),
  CHECK (tipo = 1 OR tipo = 2)
);

CREATE TABLE VotoCandidato(
  idVoto INTEGER PRIMARY KEY,
  DNI INTEGER,
  FOREIGN KEY(idVoto) REFERENCES Voto(idVoto),
  FOREIGN KEY(DNI) REFERENCES Candidato(DNI)
);

CREATE TABLE VotoConsultaPopular(
  idVoto INTEGER PRIMARY KEY,
  idConsulta INTEGER,
  FOREIGN KEY(idVoto) REFERENCES Voto(idVoto),
  FOREIGN KEY(idConsulta) REFERENCES ConsultaPopular(idConsulta)
);

CREATE TABLE ViveEn(
  DNI INTEGER,
  idTerritorio INTEGER,
  fechaInicio INTEGER,
  fechaFin INTEGER,
  PRIMARY KEY(DNI, idTerritorio, fechaInicio),
  FOREIGN KEY(DNI) REFERENCES Votante(DNI),
  FOREIGN KEY(idTerritorio) REFERENCES Territorio(idTerritorio),
  CHECK (fechaFin = -1 OR fechaFin > fechaInicio)
);

CREATE TABLE RigePara(
  idCargo INTEGER,
  idTerritorio INTEGER,
  PRIMARY KEY(idCargo, idTerritorio),
  FOREIGN KEY(idCargo) REFERENCES Cargo(idCargo),
  FOREIGN KEY(idTerritorio) REFERENCES Territorio(idTerritorio)
);

CREATE TABLE SePostulaA(
  DNI INTEGER,
  idEleccion INTEGER,
  idPartido INTEGER,
  cantVotos INTEGER,
  PRIMARY KEY(DNI, idEleccion),
  FOREIGN KEY(DNI) REFERENCES Candidato(DNI),
  FOREIGN KEY(idPartido) REFERENCES PartidoPolitico(idPartido),
  FOREIGN KEY(idEleccion) REFERENCES VotacionCandidato(idEleccion)
);

CREATE TABLE VotaEn (
  DNI INTEGER, 
  idEleccion INTEGER, 
  idMesa INTEGER,   
  PRIMARY KEY(DNI,idEleccion,idMesa), 
  FOREIGN KEY(DNI) REFERENCES Votante(DNI), 
  FOREIGN KEY(idEleccion) REFERENCES VotacionEleccion(idEleccion), 
  FOREIGN KEY(idMesa) REFERENCES Mesa(idMesa)
);

\end{lstlisting} 
\end{subsection}

\begin{subsection}{Triggers}

\begin{lstlisting}[language=SQL, basicstyle=\footnotesize]

CREATE TRIGGER AsignarVotoRandomCandidato AFTER INSERT ON Vota
  WHEN New.idEleccion IN (SELECT vc.idEleccion FROM VotacionCandidato vc)
  BEGIN       
    iNSERT INTO Voto(idVoto, idEleccion, idMesa, tipo) 
      SELECT MAX(idVoto)+1, New.idEleccion, New.idMesa, 1
      FROM Voto;

    INSERT INTO VotoCandidato (idVoto, DNI)       
      SELECT MAX(idVoto), DNI 
      FROM Voto, SePostulaA spa
      WHERE spa.idEleccion = New.idEleccion AND 
        DNI IN (SELECT DNI FROM SePostulaA ORDER BY RANDOM() LIMIT 1);
  END;

CREATE TRIGGER AsignarVotoRandomConsultaPopular AFTER INSERT ON Vota
  WHEN New.idEleccion IN (SELECT vc.idEleccion FROM VotacionConsultaPopular vc)
  BEGIN       
    iNSERT INTO Voto(idVoto, idEleccion, idMesa, tipo) 
      SELECT MAX(idVoto)+1, New.idEleccion, New.idMesa, 2 
      FROM Voto;

    INSERT INTO VotoConsultaPopular (idVoto, idConsulta) 
      SELECT MAX(idVoto), idConsulta FROM Voto, ConsultaPopular cp
      WHERE cp.idEleccion = New.idEleccion AND
        idConsulta IN (SELECT idConsulta
                       FROM ConsultaPopular ORDER BY RANDOM() LIMIT 1);
  END;

CREATE TRIGGER SumarVotos AFTER INSERT ON VotoCandidato
  BEGIN       
    UPDATE SePostulaA SET 
      cantVotos = (SELECT COUNT(*)
                   FROM VotoCandidato vc 
                   WHERE vc.DNI = SePostulaA.DNI)
      WHERE SePostulaA.DNI = New.DNI;     
  END;

CREATE TRIGGER FiscalNoPermitido BEFORE INSERT ON EsFiscal  
  BEGIN
    SELECT (RAISE(ABORT,"TRIGGER Error 'EsFiscal': 
      Ya existe un fiscal del partido politico en esa mesa."))
    WHERE EXISTS (SELECT 1 FROM EsFiscal AS ef 
      WHERE New.idPartido = ef.idPartido
    );    
  END;

CREATE TRIGGER VotoInvalido BEFORE INSERT ON Vota 
  BEGIN
    SELECT (RAISE (ABORT, "TRIGGER Error 'Vota': El votante no esta en el padron."))
    WHERE NOT EXISTS (SELECT 1 FROM VotaEn WHERE VotaEn.idEleccion = New.idEleccion
    AND VotaEn.idMesa = New.idMesa AND VotaEn.DNI = New.DNI);
  END;

\end{lstlisting} 

\end{subsection}
\end{section}

\begin{section}{Consultas}


\noindent Consulta 1: Obtener los ganadores de las elecciones transcurridas en el \'ultimo año.

\begin{lstlisting}[language=SQL, basicstyle=\footnotesize]

SELECT v.nombre
FROM Votante v, Candidato can, VotacionCandidato vc, SePostulaA spa
WHERE v.dni = can.dni
AND can.dni = spa.dni
AND vc.idEleccion IN (SELECT vc2.idEleccion
                      FROM VotacionEleccion ve, VotacionCandidato vc2
                      WHERE ve.idEleccion = vc2.idEleccion
                      AND fecha >= ( SELECT MAX(fecha)
                                     FROM VotacionEleccion))
AND vc.idEleccion = spa.idEleccion
AND spa.cantVotos = (SELECT MAX(spa2.cantVotos)
                     FROM SePostulaA spa2
                     WHERE spa2.idEleccion = vc.idEleccion);

Esta consulta toma los candidatos que participan de las \'ultimas elecciones a partir de la VotacionCandidato y luego los filtra de acuerdo con la cantidad de votos, para as\'i obtener a los ganadores.

\end{lstlisting} 

\noindent Consulta 2: Consultar las 5 personas que m\'as tarde fueron a votar antes de terminar la votaci\'on por cada centro electoral en na elecci\'on.

\begin{lstlisting}[language=SQL, basicstyle=\footnotesize]

SELECT cen.direccion, vot.dni
FROM Vota v, Votante vot, Mesa m, Centro cen
WHERE v.idMesa = m.idMesa
AND vot.dni = v.dni
AND m.idCentro = cen.idCentro
AND v.idEleccion IN (SELECT ve.idEleccion
                     FROM VotacionEleccion ve
                     WHERE fecha >= (SELECT MAX(fecha)
                                     FROM VotacionEleccion))
AND vot.dni IN (SELECT vot1.dni
                FROM Votante vot1, Vota v1, Mesa m1
                WHERE v1.idMesa = m1.idMesa
                AND m1.idCentro = cen.idCentro
                AND vot1.dni = v1.dni
                ORDER BY hora DESC
                LIMIT 5)
ORDER BY direccion,hora DESC;

Aqu\'i nos basamos en la \'ultima elecci\'on para seleccionar a los votantes que participaronde ella y luego tomamos \'unicamente aquellos que en horario llegaron m\'as tarde. El ORDER BY hora DESC sumado al LIMIT 5 del SELECT anidado nos permite restringir los DNIs de los votantes por horario m\'as tard\'io. Luego, agrupamos las tuplas de salida por centro para clarificar el resultado.

\end{lstlisting} 

\noindent Consulta 3: Consultar quienes fueron los partidos pol\'iticos que obtuvieron m\'as del 20$\%$ en las \'ultimas 5 elecciones provinciales a gobernador.



\begin{lstlisting}[language=SQL, basicstyle=\footnotesize]

SELECT pp.nombre
FROM PartidoPolitico pp, 
     SePostulaA spa, 
     Candidato can, 
     Cargo c, 
     VotacionCandidato vc
WHERE pp.idPartido = spa.idPartido
AND can.dni = spa.dni
AND spa.idEleccion = vc.idEleccion
AND vc.idEleccion IN (SELECT ve.idEleccion
                      FROM VotacionCandidato vc2, 
                           VotacionEleccion ve, Cargo c2
                      WHERE vc2.idEleccion = ve.idEleccion
                      AND c2.idCargo = vc2.idCargo
                      AND c2.nombre LIKE 'Gobernador'
                      ORDER BY ve.fecha DESC
                      LIMIT 5)
AND vc.idCargo = c.idCargo
AND c.nombre LIKE 'Gobernador'
AND spa.cantVotos > (SELECT SUM(spa2.cantVotos)*0.2
                     FROM SePostulaA spa2
                     WHERE spa2.idEleccion = vc.idEleccion);

En este caso tomamos los partidos pol\'iticos de los candidatos que participan en las \'ultimas cinco elecci\'ones de manera similar a como tomamos los votantes m\'as tard\'ios de la consulta anterior, filtrando las elecciones por cargo a Gobernador. Luego pedimos que dichos candidatos junten una cantidad de votos mayor al 20% de la suma total de votos en la elecci\'on de la cual participan.

\end{lstlisting} 


\end{section}
