\begin{section}{Modelo Entidad Relaci\'on}

\begin{subsection}{Diagrama de entidad relaci\'on}

%diagramita del DER

\end{subsection}

\begin{subsection}{Aclaraciones en lenguaje natural}

\begin{itemize}

\item La cantidad de los votos de una elecci\'on tiene que ser igual a la sumatoria de los votos obtenidos por todos los candidatos en esa elecci\'on.

\item La cantidad de votos debe ser menor o igual a la cantidad de ciudadanos en la sumatoria de los padrones de las mesas de esa elecci\'on.

\item (*) Se deben cumplir que existan los cargos de ``presidente", ``Vicepresidente'' y ``t\'ecico" por cada mesa en cada elecci\'on.

\item (*) Un mismpo ciudadano no puede ocupar m\'as de un cargo por elecci\'on (Presidente, vicepresidente, t\'ecnico o fiscal).

\item Las fechas de la relaci\'on \textit{voto} deben coincidir con alguna fecha de elecci\'on.

\item Las fechas de la relaci\'on \textit{participa} deben coincidir con alguna fecha de elecci\'on.

\item Un ciudadano no puede votar m\'as de una vez por elecci\'on.

\item Un ciudadano solo puede votar en la mesa en la que est\'e empadronado para la elecci\'on.

\end{itemize}

\end{subsection}

\end{section}
