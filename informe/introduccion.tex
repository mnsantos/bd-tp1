\begin{section}{Introduccion}
Este trabajo consiste en el modelado e implementación de una base de datos correspondiente a un problema del mundo real: realización de un sistema de voto electrónico.

El objetivo es contar con una base de datos que soporte votaciones que abarquen todo el territorio nacional, es decir de cualquier cargo público de cada territorio que compone la nación. El sistema debe poder guardar informaci\'on sobre todos los actores involucrados: votantes, candidatos, fiscales de mesa, etc y sus atribuciones y acciones dentro del sistema. Además debe permitir obtener los resultados de las votaciones.
\end{section}