\begin{section}{Conclusiones}
Luego de haber desarrollado un primer diagrama de entidad relación encontramos que habían consultas que no podíamos hacer porque nos faltaba información o que resultaban muy complejas. Tuvimos que volver varias veces sobre el modelo y rediseñarlo hasta dar con una solución que nos permitiera realizar todas las consultas necesarias. 

Un buen diseño previo siempre es muy importante, pero la falta de experiencia, nos demostró que por más buen diseño que pensamos que habíamos logrado, luego presentaba problemas que no habíamos contemplado.

Otro punto importante a la hora de modelizar fue comprender la totalidad del enunciado y abstraer solo la información importante y necesaria. Esto es lo que sucede cuando se quiere pasar de un problema del mundo real a un modelo computacional. En el caso particular de este trabajo práctico puede ser que sea más fácil que un problema común del mundo real, pues se nos presentó una serie de reglas claras, con pocas ambigüedades. 

Nos resultó a veces difícil encontrar un punto de corte para la información relevante para el modelo. Dado que se está modelando un problema del mundo real, uno se puede explayar tanto como le parezca necesario e incluir mucha información que tal vez no hace al problema en cuestión.

Por último, consideramos que este trabajo resulta una buena manera de introducirse en la práctica de diseño de base de datos, al ser un problema actual sacado del mundo real, y que permite distintos enfoques para su resolución.
\end{section}